\section{Evaluation}
\label{sec:evaluation}

\input{tab-input-graphs-condensed}

\begin{figure}[htb]
    \begin{subfigure}[t]{\linewidth}
        \centering
        \includegraphics[width=\linewidth]{reporting/results-load}
        \caption{Load times of the solutions.}
        \label{fig:load-times}
    \end{subfigure}
    \begin{subfigure}[t]{\linewidth}
        \centering
        \includegraphics[width=\linewidth]{reporting/results-computation}
        \caption{Computation times of the solutions.}
        \label{fig:computation-times}
    \end{subfigure}
    \caption{Execution times of queries.}
    \label{fig:execution-times}
    \vspace{-3ex}
\end{figure}

% \begin{figure}[htb]
%     \centering
%     \includegraphics[width=\linewidth]{reporting/results-computation}
%     \caption{Computation times of the solutions.}
%     \label{fig:computation-times}
%     %\vspace{-3ex}
% \end{figure}


\subsection{Benchmark Setup}

\paragraph{Goal}
%
We designed an experiment to compare the performance and scalability of our implementation against the top solutions.

\paragraph{Solutions}
%
We implemented our solution in C++ using version v3.3.3 of \gxb and the latest \lagraph library~\cite{DBLP:conf/ipps/MattsonDKBMMY19}.\footnote{\url{https://github.com/ldbc/sigmod2014-pc-graphblas}}
We compared our implementation against two solutions of the programming contest created by teams ``AWFY'' (ranked 1st) and ``blxlrsmb'' (ranked 4th), updated for GCC~9.\footnote{\url{https://github.com/ftsrg/sigmod2014-pc-top-solutions}}

% AWFY -> TU Munich
% blxlrsmb -> Tsinghua University
% We did not contact their original authors for further improvements, see "Threats to validity"

\paragraph{Data sets}
%
We were unable to obtain the data sets used in the contest, therefore we generated similar ones using the LDBC SNB Datagen's~\cite{DBLP:conf/tpctc/PhamBE12} 2014 version.\footnote{\url{https://github.com/ldbc/ldbc_snb_datagen/releases/tag/early2014}} The data set statistics are shown in \autoref{tab:input-graphs}.
We also implemented a parameter generator that produces query input parameters using uniform sampling.

\paragraph{Environment}
%
We performed the experiments on a cloud virtual machine with 32 (logical)  Intel Xeon Skylake CPU cores clocked at 2GHz, 120GB RAM, and SSD storage, running Ubuntu 20.04. We used the GCC~9.3.0 compiler.
We ran each benchmark with 80~different parameters.

\subsection{Analysis}

\paragraph{Performance}
%
We visualized the distribution of the execution times obtained during the experiments:
load times are shown in \autoref{fig:load-times} and
computation times in \autoref{fig:computation-times}.
For \emph{load times}, other solutions consistently outperform our solution. AWFY provides particularly fast loads (which can be attributed to its usage of advanced CSV loading techniques presented in~\cite{DBLP:journals/pvldb/MuhlbauerRSRK013}).
However, for \emph{computation times}, our solution is competitive for Q2 (often outperforming other solutions). It provides good performance for Q3 and Q4, staying within an order of magnitude compared to the other highly-optimized solutions.
Our solution exhibits a bimodal distribution for Q1, which can be attributed to the configuration parameter $x$: for $x > -1$, computing the induced subgraph is expensive.
The execution times of our solution for Q1 are noticeably longer than the competition's due to precomputing the entire induced graph (instead of computing the relevant edges on-the-fly during traversal).

\paragraph{Conciseness}
%
We characterized the conciseness of each solution using lines of C++ code: %.All solutions were implemented in C++:
AWFY consisted of 9,800 lines, 
blxlrsmb used 6,500 lines,
while our code used 3,500 lines.

% =============== AWFY ===============
% sigmod14-pc-top-solutions/AWFY$ cloc --exclude-dir include .
%      131 text files.
%      103 unique files.                                          
%      414 files ignored.http://cloc.sourceforge.net v 1.60  T=0.50 s (104.0 files/s, 21166.1 lines/s)
% -------------------------------------------------------------------------------
% Language                     files          blank        comment           code
% -------------------------------------------------------------------------------
% C++                             17            952            495           4549
% C                                2            226            100           1010
% make                             3            428            197            924
% CMake                           20            154             75            856
% C/C++ Header                     5             99             67            296
% XML                              4              0              0            149
% Bourne Shell                     1              0              0              1
% -------------------------------------------------------------------------------
% SUM:                            52           1859            934           7785
% -------------------------------------------------------------------------------

% include/ folder:  exclude: grep -L "Moritz Kaufmann" `find include -type f`
%                   include: ag -l "Moritz Kaufmann" include/

% cloc `ag -l "Moritz Kaufmann" include/`
%       27 text files.
%       27 unique files.                              
%        0 files ignored.

% http://cloc.sourceforge.net v 1.60  T=0.03 s (823.7 files/s, 158708.5 lines/s)
% -------------------------------------------------------------------------------
% Language                     files          blank        comment           code
% -------------------------------------------------------------------------------
% C/C++ Header                    27            815            475           3912
% -------------------------------------------------------------------------------
% SUM:                            27            815            475           3912
% -------------------------------------------------------------------------------

% C/C++/header: 4549+1010+296+3912=9767 ~ 9800
% ^^^^^^^^^^^^^^^ AWFY ^^^^^^^^^^^^^^^

\paragraph{Threats to validity}
%
We remark that the solutions used in the experiments are 6+ years old and might be improved by further optimizations that were unavailable at the time of the contest.
We have updated the solutions to GCC~9 but did not apply any further optimizations nor did we contact their original authors.

%The programming contest required participants to solve a workload containing a mix of multiple queries. Therefore, solutions might have opted to create single-threaded implementations of simpler queries (while using the rest of the threads for the more complex queries).
%Additionally, we had access to the optimizations of all teams (in the form of posters, presentations, and papers) that authors did not have access to during the contest.
