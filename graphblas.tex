\section{The GraphBLAS}
\label{sec:grb}

\input{table-graphblas-notation}

\paragraph{Goal}
%
The goal of \grb is to create a layer of abstraction between the graph algorithms and the graph analytics framework, separating the concerns of the algorithm developers from those of the framework developers and hardware designers.
To achieve this, it builds on the theoretical framework of matrix operations on arbitrary semirings~\cite{DBLP:conf/hpec/KepnerABBFGHKLM16}, which allows defining graph algorithms in the language of linear algebra~\cite{DBLP:books/siam/11/KG2011}.
To ensure portability, the \grb standard defines a C API that can be implemented on a variety of hardware including GPUs.

\paragraph{Data structures}
%
A graph with $n$ vertices can be stored as a square adjacency matrix $\grbm{A} \in \grbuint^{n \times n}$, where rows and columns both represent vertices of the graph and element~$ A(i,j) $ contains the number of edges from vertex~$i$ to vertex~$j$. If the graph is undirected, the matrix is symmetric.

\paragraph{Navigation}
%
The fundamental step in \grb is the multiplication of an adjacency matrix with another matrix or vector over a selected semiring.
For example, 
the operation $\grbm{HasMember} \grblorland \grbm{IsLocatedIn}$ computed over the \emph{``logical or.logical and''} semiring returns a matrix representing the Places where a Forum's members are located in.
Meanwhile, when computed over the conventional arithmetic \emph{``plus.times''} semiring,
$\grbm{HasMember} \grbplustimes \grbm{IsLocatedIn}$  also returns the number of such Persons.
A traversal from a certain set of vertices can be expressed by using a boolean vector $\grbv{f}$ (often referred to as the \emph{frontier}, \emph{wavefront}, or \emph{queue})
and setting \texttt{true} values for the elements corresponding to source vertices. For example, for Forums $\grbv{f} \in \grbbool^{\grbcnt{forums}}$,
$\grbv{f} \grblorland \grbm{HasMember}$
returns the Persons who belong to any of the forums in $\grbv{f}$.
The BFS navigation step can also be captured using other semirings such as
$\grbsemiringops{lor}{first}$, where $\grboperator{first}(x, y) = x$;
$\grbsemiringops{lor}{second}$, where $\grboperator{second}(x, y) = y$; and
$\grbsemiringops{any}{pair}$, where $\grbany(x,y)$ returns either $x$ or $y$, and $\grbpair(x,y) = 1$~\cite{GxBUserGuide}.

\subsection{Notation}
%
\autoref{tab:graphblas-notation} contains the notation of \grb operations
Additionally, we use $\grbm{D} = \grbdiag{\grba{j}, \grbs{n}}$ to construct a diagonal matrix $\grbm{D} \grbassign \{ \grba{j}, \grba{j}, [1, 1, \ldots, 1] \}$. The elements of the matrix are $\grbm{D}(\grbs{j}, \grbs{j}) = 1$ for $\grbs{j} \in \grba{j}$.

\subsubsection{Masks}

Masks
$\grbmC \grbmask{\grbm{M}}$ and
$\grbvw \grbmask{\grbv{m}}$ are used to selectively write to the result matrix/vector.
The complements of the masks can be selected with the negation symbol, denoted with:
$\grbmC \grbmask{\grbneg \grbm{M}}$ and
$\grbvw \grbmask{\grbneg \grbv{m}}$,
respectively.

Masks
with ``replace'' semantics (annihilating all elements outside the mask)
are denoted with

\begin{itemize}
\item $\grbmC\grbmaskreplace{\grbm{M}}$
\item $\grbmC\grbmaskreplace{\grbneg \grbm{M}}$
\item $\grbvw\grbmaskreplace{\grbv{m}}$
\item $\grbvw\grbmaskreplace{\grbneg \grbv{m}}$
\end{itemize}

The structure of the mask is denoted with:

\begin{itemize}
\item $\grbmC\grbmask{\grbstr{\grbm{M}}}$
\item $\grbmC\grbmask{\grbneg \grbstr{\grbm{M}}}$
\item $\grbvw\grbmask{\grbstr{\grbv{m}}}$
\item $\grbvw\grbmask{\grbneg \grbstr{\grbv{m}}}$
\end{itemize}

Combining structure and replace semantics is possible:        

\begin{itemize}
\item $\grbmC\grbmaskreplace{\grbstr{\grbm{M}}}$
\item $\grbmC\grbmaskreplace{\grbneg \grbstr{\grbm{M}}}$
\item $\grbvw\grbmaskreplace{\grbstr{\grbv{m}}}$
\item $\grbvw\grbmaskreplace{\grbneg \grbstr{\grbv{m}}}$
\end{itemize}

Initializing scalars, vectors, and matrices (GraphBLAS methods):

\begin{itemize}
    \item $\grbnewscalar{\grbss}{\grbfloat}{64}$
    \item $\grbnewvector{\grbvw}{\grbfloat}{32}{n}$
    \item $\grbnewmatrix{\grbmA}{\grbuint}{16}{m}{n}$
    \item $\grbnewmatrix{\grbmA}{\grbint}{64}{k}{m}$
\end{itemize}
